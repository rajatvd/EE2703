
% Default to the notebook output style

    


% Inherit from the specified cell style.




    
\documentclass[11pt]{article}

    
    
    \usepackage[T1]{fontenc}
    % Nicer default font (+ math font) than Computer Modern for most use cases
    \usepackage{mathpazo}

    % Basic figure setup, for now with no caption control since it's done
    % automatically by Pandoc (which extracts ![](path) syntax from Markdown).
    \usepackage{graphicx}
    % We will generate all images so they have a width \maxwidth. This means
    % that they will get their normal width if they fit onto the page, but
    % are scaled down if they would overflow the margins.
    \makeatletter
    \def\maxwidth{\ifdim\Gin@nat@width>\linewidth\linewidth
    \else\Gin@nat@width\fi}
    \makeatother
    \let\Oldincludegraphics\includegraphics
    % Set max figure width to be 80% of text width, for now hardcoded.
    \renewcommand{\includegraphics}[1]{\Oldincludegraphics[width=.8\maxwidth]{#1}}
    % Ensure that by default, figures have no caption (until we provide a
    % proper Figure object with a Caption API and a way to capture that
    % in the conversion process - todo).
    \usepackage{caption}
    \DeclareCaptionLabelFormat{nolabel}{}
    \captionsetup{labelformat=nolabel}

    \usepackage{adjustbox} % Used to constrain images to a maximum size 
    \usepackage{xcolor} % Allow colors to be defined
    \usepackage{enumerate} % Needed for markdown enumerations to work
    \usepackage{geometry} % Used to adjust the document margins
    \usepackage{amsmath} % Equations
    \usepackage{amssymb} % Equations
    \usepackage{textcomp} % defines textquotesingle
    % Hack from http://tex.stackexchange.com/a/47451/13684:
    \AtBeginDocument{%
        \def\PYZsq{\textquotesingle}% Upright quotes in Pygmentized code
    }
    \usepackage{upquote} % Upright quotes for verbatim code
    \usepackage{eurosym} % defines \euro
    \usepackage[mathletters]{ucs} % Extended unicode (utf-8) support
    \usepackage[utf8x]{inputenc} % Allow utf-8 characters in the tex document
    \usepackage{fancyvrb} % verbatim replacement that allows latex
    \usepackage{grffile} % extends the file name processing of package graphics 
                         % to support a larger range 
    % The hyperref package gives us a pdf with properly built
    % internal navigation ('pdf bookmarks' for the table of contents,
    % internal cross-reference links, web links for URLs, etc.)
    \usepackage{hyperref}
    \usepackage{longtable} % longtable support required by pandoc >1.10
    \usepackage{booktabs}  % table support for pandoc > 1.12.2
    \usepackage[inline]{enumitem} % IRkernel/repr support (it uses the enumerate* environment)
    \usepackage[normalem]{ulem} % ulem is needed to support strikethroughs (\sout)
                                % normalem makes italics be italics, not underlines
    

    
    
    % Colors for the hyperref package
    \definecolor{urlcolor}{rgb}{0,.145,.698}
    \definecolor{linkcolor}{rgb}{.71,0.21,0.01}
    \definecolor{citecolor}{rgb}{.12,.54,.11}

    % ANSI colors
    \definecolor{ansi-black}{HTML}{3E424D}
    \definecolor{ansi-black-intense}{HTML}{282C36}
    \definecolor{ansi-red}{HTML}{E75C58}
    \definecolor{ansi-red-intense}{HTML}{B22B31}
    \definecolor{ansi-green}{HTML}{00A250}
    \definecolor{ansi-green-intense}{HTML}{007427}
    \definecolor{ansi-yellow}{HTML}{DDB62B}
    \definecolor{ansi-yellow-intense}{HTML}{B27D12}
    \definecolor{ansi-blue}{HTML}{208FFB}
    \definecolor{ansi-blue-intense}{HTML}{0065CA}
    \definecolor{ansi-magenta}{HTML}{D160C4}
    \definecolor{ansi-magenta-intense}{HTML}{A03196}
    \definecolor{ansi-cyan}{HTML}{60C6C8}
    \definecolor{ansi-cyan-intense}{HTML}{258F8F}
    \definecolor{ansi-white}{HTML}{C5C1B4}
    \definecolor{ansi-white-intense}{HTML}{A1A6B2}

    % commands and environments needed by pandoc snippets
    % extracted from the output of `pandoc -s`
    \providecommand{\tightlist}{%
      \setlength{\itemsep}{0pt}\setlength{\parskip}{0pt}}
    \DefineVerbatimEnvironment{Highlighting}{Verbatim}{commandchars=\\\{\}}
    % Add ',fontsize=\small' for more characters per line
    \newenvironment{Shaded}{}{}
    \newcommand{\KeywordTok}[1]{\textcolor[rgb]{0.00,0.44,0.13}{\textbf{{#1}}}}
    \newcommand{\DataTypeTok}[1]{\textcolor[rgb]{0.56,0.13,0.00}{{#1}}}
    \newcommand{\DecValTok}[1]{\textcolor[rgb]{0.25,0.63,0.44}{{#1}}}
    \newcommand{\BaseNTok}[1]{\textcolor[rgb]{0.25,0.63,0.44}{{#1}}}
    \newcommand{\FloatTok}[1]{\textcolor[rgb]{0.25,0.63,0.44}{{#1}}}
    \newcommand{\CharTok}[1]{\textcolor[rgb]{0.25,0.44,0.63}{{#1}}}
    \newcommand{\StringTok}[1]{\textcolor[rgb]{0.25,0.44,0.63}{{#1}}}
    \newcommand{\CommentTok}[1]{\textcolor[rgb]{0.38,0.63,0.69}{\textit{{#1}}}}
    \newcommand{\OtherTok}[1]{\textcolor[rgb]{0.00,0.44,0.13}{{#1}}}
    \newcommand{\AlertTok}[1]{\textcolor[rgb]{1.00,0.00,0.00}{\textbf{{#1}}}}
    \newcommand{\FunctionTok}[1]{\textcolor[rgb]{0.02,0.16,0.49}{{#1}}}
    \newcommand{\RegionMarkerTok}[1]{{#1}}
    \newcommand{\ErrorTok}[1]{\textcolor[rgb]{1.00,0.00,0.00}{\textbf{{#1}}}}
    \newcommand{\NormalTok}[1]{{#1}}
    
    % Additional commands for more recent versions of Pandoc
    \newcommand{\ConstantTok}[1]{\textcolor[rgb]{0.53,0.00,0.00}{{#1}}}
    \newcommand{\SpecialCharTok}[1]{\textcolor[rgb]{0.25,0.44,0.63}{{#1}}}
    \newcommand{\VerbatimStringTok}[1]{\textcolor[rgb]{0.25,0.44,0.63}{{#1}}}
    \newcommand{\SpecialStringTok}[1]{\textcolor[rgb]{0.73,0.40,0.53}{{#1}}}
    \newcommand{\ImportTok}[1]{{#1}}
    \newcommand{\DocumentationTok}[1]{\textcolor[rgb]{0.73,0.13,0.13}{\textit{{#1}}}}
    \newcommand{\AnnotationTok}[1]{\textcolor[rgb]{0.38,0.63,0.69}{\textbf{\textit{{#1}}}}}
    \newcommand{\CommentVarTok}[1]{\textcolor[rgb]{0.38,0.63,0.69}{\textbf{\textit{{#1}}}}}
    \newcommand{\VariableTok}[1]{\textcolor[rgb]{0.10,0.09,0.49}{{#1}}}
    \newcommand{\ControlFlowTok}[1]{\textcolor[rgb]{0.00,0.44,0.13}{\textbf{{#1}}}}
    \newcommand{\OperatorTok}[1]{\textcolor[rgb]{0.40,0.40,0.40}{{#1}}}
    \newcommand{\BuiltInTok}[1]{{#1}}
    \newcommand{\ExtensionTok}[1]{{#1}}
    \newcommand{\PreprocessorTok}[1]{\textcolor[rgb]{0.74,0.48,0.00}{{#1}}}
    \newcommand{\AttributeTok}[1]{\textcolor[rgb]{0.49,0.56,0.16}{{#1}}}
    \newcommand{\InformationTok}[1]{\textcolor[rgb]{0.38,0.63,0.69}{\textbf{\textit{{#1}}}}}
    \newcommand{\WarningTok}[1]{\textcolor[rgb]{0.38,0.63,0.69}{\textbf{\textit{{#1}}}}}
    
    
    % Define a nice break command that doesn't care if a line doesn't already
    % exist.
    \def\br{\hspace*{\fill} \\* }
    % Math Jax compatability definitions
    \def\gt{>}
    \def\lt{<}
    % Document parameters
    \title{EE2703 Applied Programming Lab - Assignment 2}
		
		\author{Rajat Vadiraj Dwaraknath EE16B033}
    
    
    

    % Pygments definitions
    
\makeatletter
\def\PY@reset{\let\PY@it=\relax \let\PY@bf=\relax%
    \let\PY@ul=\relax \let\PY@tc=\relax%
    \let\PY@bc=\relax \let\PY@ff=\relax}
\def\PY@tok#1{\csname PY@tok@#1\endcsname}
\def\PY@toks#1+{\ifx\relax#1\empty\else%
    \PY@tok{#1}\expandafter\PY@toks\fi}
\def\PY@do#1{\PY@bc{\PY@tc{\PY@ul{%
    \PY@it{\PY@bf{\PY@ff{#1}}}}}}}
\def\PY#1#2{\PY@reset\PY@toks#1+\relax+\PY@do{#2}}

\expandafter\def\csname PY@tok@sb\endcsname{\def\PY@tc##1{\textcolor[rgb]{0.73,0.13,0.13}{##1}}}
\expandafter\def\csname PY@tok@kn\endcsname{\let\PY@bf=\textbf\def\PY@tc##1{\textcolor[rgb]{0.00,0.50,0.00}{##1}}}
\expandafter\def\csname PY@tok@ch\endcsname{\let\PY@it=\textit\def\PY@tc##1{\textcolor[rgb]{0.25,0.50,0.50}{##1}}}
\expandafter\def\csname PY@tok@nl\endcsname{\def\PY@tc##1{\textcolor[rgb]{0.63,0.63,0.00}{##1}}}
\expandafter\def\csname PY@tok@kp\endcsname{\def\PY@tc##1{\textcolor[rgb]{0.00,0.50,0.00}{##1}}}
\expandafter\def\csname PY@tok@nf\endcsname{\def\PY@tc##1{\textcolor[rgb]{0.00,0.00,1.00}{##1}}}
\expandafter\def\csname PY@tok@kr\endcsname{\let\PY@bf=\textbf\def\PY@tc##1{\textcolor[rgb]{0.00,0.50,0.00}{##1}}}
\expandafter\def\csname PY@tok@c1\endcsname{\let\PY@it=\textit\def\PY@tc##1{\textcolor[rgb]{0.25,0.50,0.50}{##1}}}
\expandafter\def\csname PY@tok@k\endcsname{\let\PY@bf=\textbf\def\PY@tc##1{\textcolor[rgb]{0.00,0.50,0.00}{##1}}}
\expandafter\def\csname PY@tok@s2\endcsname{\def\PY@tc##1{\textcolor[rgb]{0.73,0.13,0.13}{##1}}}
\expandafter\def\csname PY@tok@gu\endcsname{\let\PY@bf=\textbf\def\PY@tc##1{\textcolor[rgb]{0.50,0.00,0.50}{##1}}}
\expandafter\def\csname PY@tok@cp\endcsname{\def\PY@tc##1{\textcolor[rgb]{0.74,0.48,0.00}{##1}}}
\expandafter\def\csname PY@tok@mh\endcsname{\def\PY@tc##1{\textcolor[rgb]{0.40,0.40,0.40}{##1}}}
\expandafter\def\csname PY@tok@vg\endcsname{\def\PY@tc##1{\textcolor[rgb]{0.10,0.09,0.49}{##1}}}
\expandafter\def\csname PY@tok@mi\endcsname{\def\PY@tc##1{\textcolor[rgb]{0.40,0.40,0.40}{##1}}}
\expandafter\def\csname PY@tok@vc\endcsname{\def\PY@tc##1{\textcolor[rgb]{0.10,0.09,0.49}{##1}}}
\expandafter\def\csname PY@tok@cs\endcsname{\let\PY@it=\textit\def\PY@tc##1{\textcolor[rgb]{0.25,0.50,0.50}{##1}}}
\expandafter\def\csname PY@tok@gr\endcsname{\def\PY@tc##1{\textcolor[rgb]{1.00,0.00,0.00}{##1}}}
\expandafter\def\csname PY@tok@dl\endcsname{\def\PY@tc##1{\textcolor[rgb]{0.73,0.13,0.13}{##1}}}
\expandafter\def\csname PY@tok@na\endcsname{\def\PY@tc##1{\textcolor[rgb]{0.49,0.56,0.16}{##1}}}
\expandafter\def\csname PY@tok@nb\endcsname{\def\PY@tc##1{\textcolor[rgb]{0.00,0.50,0.00}{##1}}}
\expandafter\def\csname PY@tok@cpf\endcsname{\let\PY@it=\textit\def\PY@tc##1{\textcolor[rgb]{0.25,0.50,0.50}{##1}}}
\expandafter\def\csname PY@tok@si\endcsname{\let\PY@bf=\textbf\def\PY@tc##1{\textcolor[rgb]{0.73,0.40,0.53}{##1}}}
\expandafter\def\csname PY@tok@o\endcsname{\def\PY@tc##1{\textcolor[rgb]{0.40,0.40,0.40}{##1}}}
\expandafter\def\csname PY@tok@gt\endcsname{\def\PY@tc##1{\textcolor[rgb]{0.00,0.27,0.87}{##1}}}
\expandafter\def\csname PY@tok@ne\endcsname{\let\PY@bf=\textbf\def\PY@tc##1{\textcolor[rgb]{0.82,0.25,0.23}{##1}}}
\expandafter\def\csname PY@tok@nn\endcsname{\let\PY@bf=\textbf\def\PY@tc##1{\textcolor[rgb]{0.00,0.00,1.00}{##1}}}
\expandafter\def\csname PY@tok@w\endcsname{\def\PY@tc##1{\textcolor[rgb]{0.73,0.73,0.73}{##1}}}
\expandafter\def\csname PY@tok@nt\endcsname{\let\PY@bf=\textbf\def\PY@tc##1{\textcolor[rgb]{0.00,0.50,0.00}{##1}}}
\expandafter\def\csname PY@tok@s\endcsname{\def\PY@tc##1{\textcolor[rgb]{0.73,0.13,0.13}{##1}}}
\expandafter\def\csname PY@tok@s1\endcsname{\def\PY@tc##1{\textcolor[rgb]{0.73,0.13,0.13}{##1}}}
\expandafter\def\csname PY@tok@kc\endcsname{\let\PY@bf=\textbf\def\PY@tc##1{\textcolor[rgb]{0.00,0.50,0.00}{##1}}}
\expandafter\def\csname PY@tok@nd\endcsname{\def\PY@tc##1{\textcolor[rgb]{0.67,0.13,1.00}{##1}}}
\expandafter\def\csname PY@tok@gh\endcsname{\let\PY@bf=\textbf\def\PY@tc##1{\textcolor[rgb]{0.00,0.00,0.50}{##1}}}
\expandafter\def\csname PY@tok@mo\endcsname{\def\PY@tc##1{\textcolor[rgb]{0.40,0.40,0.40}{##1}}}
\expandafter\def\csname PY@tok@err\endcsname{\def\PY@bc##1{\setlength{\fboxsep}{0pt}\fcolorbox[rgb]{1.00,0.00,0.00}{1,1,1}{\strut ##1}}}
\expandafter\def\csname PY@tok@bp\endcsname{\def\PY@tc##1{\textcolor[rgb]{0.00,0.50,0.00}{##1}}}
\expandafter\def\csname PY@tok@nc\endcsname{\let\PY@bf=\textbf\def\PY@tc##1{\textcolor[rgb]{0.00,0.00,1.00}{##1}}}
\expandafter\def\csname PY@tok@mb\endcsname{\def\PY@tc##1{\textcolor[rgb]{0.40,0.40,0.40}{##1}}}
\expandafter\def\csname PY@tok@kt\endcsname{\def\PY@tc##1{\textcolor[rgb]{0.69,0.00,0.25}{##1}}}
\expandafter\def\csname PY@tok@gi\endcsname{\def\PY@tc##1{\textcolor[rgb]{0.00,0.63,0.00}{##1}}}
\expandafter\def\csname PY@tok@ge\endcsname{\let\PY@it=\textit}
\expandafter\def\csname PY@tok@sc\endcsname{\def\PY@tc##1{\textcolor[rgb]{0.73,0.13,0.13}{##1}}}
\expandafter\def\csname PY@tok@sh\endcsname{\def\PY@tc##1{\textcolor[rgb]{0.73,0.13,0.13}{##1}}}
\expandafter\def\csname PY@tok@il\endcsname{\def\PY@tc##1{\textcolor[rgb]{0.40,0.40,0.40}{##1}}}
\expandafter\def\csname PY@tok@ss\endcsname{\def\PY@tc##1{\textcolor[rgb]{0.10,0.09,0.49}{##1}}}
\expandafter\def\csname PY@tok@go\endcsname{\def\PY@tc##1{\textcolor[rgb]{0.53,0.53,0.53}{##1}}}
\expandafter\def\csname PY@tok@sr\endcsname{\def\PY@tc##1{\textcolor[rgb]{0.73,0.40,0.53}{##1}}}
\expandafter\def\csname PY@tok@ow\endcsname{\let\PY@bf=\textbf\def\PY@tc##1{\textcolor[rgb]{0.67,0.13,1.00}{##1}}}
\expandafter\def\csname PY@tok@vi\endcsname{\def\PY@tc##1{\textcolor[rgb]{0.10,0.09,0.49}{##1}}}
\expandafter\def\csname PY@tok@se\endcsname{\let\PY@bf=\textbf\def\PY@tc##1{\textcolor[rgb]{0.73,0.40,0.13}{##1}}}
\expandafter\def\csname PY@tok@nv\endcsname{\def\PY@tc##1{\textcolor[rgb]{0.10,0.09,0.49}{##1}}}
\expandafter\def\csname PY@tok@m\endcsname{\def\PY@tc##1{\textcolor[rgb]{0.40,0.40,0.40}{##1}}}
\expandafter\def\csname PY@tok@sx\endcsname{\def\PY@tc##1{\textcolor[rgb]{0.00,0.50,0.00}{##1}}}
\expandafter\def\csname PY@tok@mf\endcsname{\def\PY@tc##1{\textcolor[rgb]{0.40,0.40,0.40}{##1}}}
\expandafter\def\csname PY@tok@sd\endcsname{\let\PY@it=\textit\def\PY@tc##1{\textcolor[rgb]{0.73,0.13,0.13}{##1}}}
\expandafter\def\csname PY@tok@gp\endcsname{\let\PY@bf=\textbf\def\PY@tc##1{\textcolor[rgb]{0.00,0.00,0.50}{##1}}}
\expandafter\def\csname PY@tok@gs\endcsname{\let\PY@bf=\textbf}
\expandafter\def\csname PY@tok@sa\endcsname{\def\PY@tc##1{\textcolor[rgb]{0.73,0.13,0.13}{##1}}}
\expandafter\def\csname PY@tok@no\endcsname{\def\PY@tc##1{\textcolor[rgb]{0.53,0.00,0.00}{##1}}}
\expandafter\def\csname PY@tok@cm\endcsname{\let\PY@it=\textit\def\PY@tc##1{\textcolor[rgb]{0.25,0.50,0.50}{##1}}}
\expandafter\def\csname PY@tok@ni\endcsname{\let\PY@bf=\textbf\def\PY@tc##1{\textcolor[rgb]{0.60,0.60,0.60}{##1}}}
\expandafter\def\csname PY@tok@fm\endcsname{\def\PY@tc##1{\textcolor[rgb]{0.00,0.00,1.00}{##1}}}
\expandafter\def\csname PY@tok@c\endcsname{\let\PY@it=\textit\def\PY@tc##1{\textcolor[rgb]{0.25,0.50,0.50}{##1}}}
\expandafter\def\csname PY@tok@kd\endcsname{\let\PY@bf=\textbf\def\PY@tc##1{\textcolor[rgb]{0.00,0.50,0.00}{##1}}}
\expandafter\def\csname PY@tok@gd\endcsname{\def\PY@tc##1{\textcolor[rgb]{0.63,0.00,0.00}{##1}}}
\expandafter\def\csname PY@tok@vm\endcsname{\def\PY@tc##1{\textcolor[rgb]{0.10,0.09,0.49}{##1}}}

\def\PYZbs{\char`\\}
\def\PYZus{\char`\_}
\def\PYZob{\char`\{}
\def\PYZcb{\char`\}}
\def\PYZca{\char`\^}
\def\PYZam{\char`\&}
\def\PYZlt{\char`\<}
\def\PYZgt{\char`\>}
\def\PYZsh{\char`\#}
\def\PYZpc{\char`\%}
\def\PYZdl{\char`\$}
\def\PYZhy{\char`\-}
\def\PYZsq{\char`\'}
\def\PYZdq{\char`\"}
\def\PYZti{\char`\~}
% for compatibility with earlier versions
\def\PYZat{@}
\def\PYZlb{[}
\def\PYZrb{]}
\makeatother


    % Exact colors from NB
    \definecolor{incolor}{rgb}{0.0, 0.0, 0.5}
    \definecolor{outcolor}{rgb}{0.545, 0.0, 0.0}



    
    % Prevent overflowing lines due to hard-to-break entities
    \sloppy 
    % Setup hyperref package
    \hypersetup{
      breaklinks=true,  % so long urls are correctly broken across lines
      colorlinks=true,
      urlcolor=urlcolor,
      linkcolor=linkcolor,
      citecolor=citecolor,
      }
    % Slightly bigger margins than the latex defaults
    
    \geometry{verbose,tmargin=1in,bmargin=1in,lmargin=1in,rmargin=1in}
    
    

    \begin{document}
    
    
    \maketitle
    
    
\section{Introduction}
In this assignment, different methods of numerical integration are explored and analyzed. First, we look at the standard scipy method for integration, namely quad, and analyze its error. This is done on a test function $$f(t) = \frac{1}{1+t^2}$$

Next, we implement our own numerical integration technique using the trapezoidal rule. The algorithm is implemented using native python loops, as well as vectorized numpy code. These two approaches are compared and contrasted. The error in this particular method is also estimated and compared with the actual error. The variation of the error with step-size is also studied, and the numerical results are compared with analytical expectations.


\section{Part 1}\label{part-1}

Numpy and matplotlib are imported inline using pylab. The plot size and
font size are increased.

    \begin{Verbatim}[commandchars=\\\{\}]
{\color{incolor}In [{\color{incolor}1}]:} \PY{c+c1}{\PYZsh{} Importing numpy and matplotlib}
        \PY{o}{\PYZpc{}}\PY{k}{pylab} inline
\end{Verbatim}


    \begin{Verbatim}[commandchars=\\\{\}]
Populating the interactive namespace from numpy and matplotlib

    \end{Verbatim}

    \begin{Verbatim}[commandchars=\\\{\}]
{\color{incolor}In [{\color{incolor}2}]:} \PY{c+c1}{\PYZsh{} Increase figure and font size}
        \PY{n}{rcParams}\PY{p}{[}\PY{l+s+s1}{\PYZsq{}}\PY{l+s+s1}{figure.figsize}\PY{l+s+s1}{\PYZsq{}}\PY{p}{]} \PY{o}{=} \PY{l+m+mi}{12}\PY{p}{,}\PY{l+m+mi}{9}
        \PY{n}{rcParams}\PY{p}{[}\PY{l+s+s1}{\PYZsq{}}\PY{l+s+s1}{font.size}\PY{l+s+s1}{\PYZsq{}}\PY{p}{]} \PY{o}{=} \PY{l+m+mi}{25}
        \PY{n}{rcParams}\PY{p}{[}\PY{l+s+s1}{\PYZsq{}}\PY{l+s+s1}{text.usetex}\PY{l+s+s1}{\PYZsq{}}\PY{p}{]} \PY{o}{=} \PY{k+kc}{True}
\end{Verbatim}


    The required function is defined below: \[f(t) = \frac{1}{1+t^2}\]

    \begin{Verbatim}[commandchars=\\\{\}]
{\color{incolor}In [{\color{incolor}3}]:} \PY{k}{def} \PY{n+nf}{f}\PY{p}{(}\PY{n}{t}\PY{p}{)}\PY{p}{:}
            \PY{k}{return} \PY{l+m+mi}{1}\PY{o}{/}\PY{p}{(}\PY{l+m+mi}{1}\PY{o}{+}\PY{n}{t}\PY{o}{*}\PY{n}{t}\PY{p}{)}
\end{Verbatim}


    \begin{Verbatim}[commandchars=\\\{\}]
{\color{incolor}In [{\color{incolor}4}]:} \PY{n}{f}\PY{p}{(}\PY{n}{arange}\PY{p}{(}\PY{l+m+mi}{10}\PY{p}{)}\PY{p}{)} \PY{c+c1}{\PYZsh{} Works with numpy arrays}
\end{Verbatim}


\begin{Verbatim}[commandchars=\\\{\}]
{\color{outcolor}Out[{\color{outcolor}4}]:} array([ 1.        ,  0.5       ,  0.2       ,  0.1       ,  0.05882353,
                0.03846154,  0.02702703,  0.02      ,  0.01538462,  0.01219512])
\end{Verbatim}
            
    \section{Part 2}\label{part-2}

A vector from \textbf{start} to \textbf{stop} with increments of size
\textbf{step} is defined below:

    \begin{Verbatim}[commandchars=\\\{\}]
{\color{incolor}In [{\color{incolor}5}]:} \PY{n}{start} \PY{o}{=} \PY{l+m+mi}{0}
        \PY{n}{end} \PY{o}{=} \PY{l+m+mi}{5}
        \PY{n}{step} \PY{o}{=} \PY{l+m+mf}{0.1}
        \PY{n}{x} \PY{o}{=} \PY{n}{linspace}\PY{p}{(}\PY{n}{start}\PY{p}{,}\PY{n}{end}\PY{p}{,}\PY{n+nb}{int}\PY{p}{(}\PY{l+m+mi}{1}\PY{o}{+}\PY{p}{(}\PY{n}{end}\PY{o}{\PYZhy{}}\PY{n}{start}\PY{p}{)}\PY{o}{/}\PY{n}{step}\PY{p}{)}\PY{p}{)}
        \PY{n}{x}
\end{Verbatim}


\begin{Verbatim}[commandchars=\\\{\}]
{\color{outcolor}Out[{\color{outcolor}5}]:} array([ 0. ,  0.1,  0.2,  0.3,  0.4,  0.5,  0.6,  0.7,  0.8,  0.9,  1. ,
                1.1,  1.2,  1.3,  1.4,  1.5,  1.6,  1.7,  1.8,  1.9,  2. ,  2.1,
                2.2,  2.3,  2.4,  2.5,  2.6,  2.7,  2.8,  2.9,  3. ,  3.1,  3.2,
                3.3,  3.4,  3.5,  3.6,  3.7,  3.8,  3.9,  4. ,  4.1,  4.2,  4.3,
                4.4,  4.5,  4.6,  4.7,  4.8,  4.9,  5. ])
\end{Verbatim}
            
    \section{Part 3}\label{part-3}

The function \textbf{f} is plotted using the vector defined above:

    \begin{Verbatim}[commandchars=\\\{\}]
{\color{incolor}In [{\color{incolor}6}]:} \PY{n}{figure}\PY{p}{(}\PY{p}{)}
        \PY{n}{title}\PY{p}{(}\PY{l+s+sa}{r}\PY{l+s+s2}{\PYZdq{}}\PY{l+s+s2}{Plot of \PYZdl{}}\PY{l+s+s2}{\PYZbs{}}\PY{l+s+s2}{frac}\PY{l+s+si}{\PYZob{}1\PYZcb{}}\PY{l+s+s2}{\PYZob{}}\PY{l+s+s2}{1+t\PYZca{}2\PYZcb{}\PYZdl{}}\PY{l+s+s2}{\PYZdq{}}\PY{p}{)}
        \PY{n}{plot}\PY{p}{(}\PY{n}{x}\PY{p}{,}\PY{n}{f}\PY{p}{(}\PY{n}{x}\PY{p}{)}\PY{p}{)}
        \PY{n}{xlabel}\PY{p}{(}\PY{l+s+s2}{\PYZdq{}}\PY{l+s+s2}{\PYZdl{}x\PYZdl{}}\PY{l+s+s2}{\PYZdq{}}\PY{p}{)}
        \PY{n}{ylabel}\PY{p}{(}\PY{l+s+s2}{\PYZdq{}}\PY{l+s+s2}{\PYZdl{}f(x)\PYZdl{}}\PY{l+s+s2}{\PYZdq{}}\PY{p}{)}
        \PY{n}{show}\PY{p}{(}\PY{p}{)}
\end{Verbatim}


    \begin{center}
    \adjustimage{max size={0.9\linewidth}{0.9\paperheight}}{Assignment2_files/Assignment2_9_0.png}
    \end{center}
    { \hspace*{\fill} \\}
    
    \section{Part 4}\label{part-4}

Finding \(tan^{-1}(x)\) by integrating \(\frac{1}{1+t^2}\).

    \begin{Verbatim}[commandchars=\\\{\}]
{\color{incolor}In [{\color{incolor}7}]:} \PY{c+c1}{\PYZsh{} Import scipy and quad}
        \PY{k+kn}{from} \PY{n+nn}{scipy}\PY{n+nn}{.}\PY{n+nn}{integrate} \PY{k}{import} \PY{n}{quad}
\end{Verbatim}


    \begin{Verbatim}[commandchars=\\\{\}]
{\color{incolor}In [{\color{incolor}8}]:} \PY{n}{integrals} \PY{o}{=} \PY{p}{[}\PY{p}{]} \PY{c+c1}{\PYZsh{} integral values}
        \PY{n}{errors} \PY{o}{=} \PY{p}{[}\PY{p}{]} \PY{c+c1}{\PYZsh{} errors returned by quad}
        
        \PY{c+c1}{\PYZsh{} Computation done using a for loop}
        \PY{k}{for} \PY{n}{a} \PY{o+ow}{in} \PY{n}{x}\PY{p}{:}
            \PY{n}{i}\PY{p}{,}\PY{n}{e} \PY{o}{=} \PY{n}{quad}\PY{p}{(}\PY{n}{f}\PY{p}{,}\PY{l+m+mi}{0}\PY{p}{,}\PY{n}{a}\PY{p}{)}
            \PY{n}{integrals}\PY{o}{.}\PY{n}{append}\PY{p}{(}\PY{n}{i}\PY{p}{)}
            \PY{n}{errors}\PY{o}{.}\PY{n}{append}\PY{p}{(}\PY{n}{e}\PY{p}{)}
\end{Verbatim}


    The plot comparing the integral method and the numpy arctan function is
shown below:

    \begin{Verbatim}[commandchars=\\\{\}]
{\color{incolor}In [{\color{incolor}9}]:} \PY{n}{fig0} \PY{o}{=} \PY{n}{figure}\PY{p}{(}\PY{p}{)}
        \PY{n}{ax} \PY{o}{=} \PY{n}{fig0}\PY{o}{.}\PY{n}{add\PYZus{}axes}\PY{p}{(}\PY{p}{[}\PY{l+m+mf}{0.05}\PY{p}{,}\PY{l+m+mf}{0.95}\PY{p}{,}\PY{l+m+mf}{0.95}\PY{p}{,}\PY{l+m+mf}{0.95}\PY{p}{]}\PY{p}{)}
        \PY{n}{ax}\PY{o}{.}\PY{n}{scatter}\PY{p}{(}\PY{n}{x}\PY{p}{,}\PY{n}{integrals}\PY{p}{,}\PY{n}{color}\PY{o}{=}\PY{l+s+s1}{\PYZsq{}}\PY{l+s+s1}{red}\PY{l+s+s1}{\PYZsq{}}\PY{p}{,}\PY{n}{s}\PY{o}{=}\PY{l+m+mi}{50}\PY{p}{)}
        \PY{n}{ax}\PY{o}{.}\PY{n}{plot}\PY{p}{(}\PY{n}{x}\PY{p}{,}\PY{n}{arctan}\PY{p}{(}\PY{n}{x}\PY{p}{)}\PY{p}{)}
        \PY{n}{ax}\PY{o}{.}\PY{n}{set\PYZus{}title}\PY{p}{(}\PY{l+s+s2}{\PYZdq{}}\PY{l+s+s2}{Comparing quad integration and numpy arctan}\PY{l+s+s2}{\PYZdq{}}\PY{p}{)}
        \PY{n}{ax}\PY{o}{.}\PY{n}{set\PYZus{}ylabel}\PY{p}{(}\PY{l+s+sa}{r}\PY{l+s+s2}{\PYZdq{}}\PY{l+s+s2}{\PYZdl{}}\PY{l+s+s2}{\PYZbs{}}\PY{l+s+s2}{int\PYZus{}0\PYZca{}a}\PY{l+s+s2}{\PYZob{}}\PY{l+s+s2}{\PYZbs{}}\PY{l+s+s2}{frac}\PY{l+s+si}{\PYZob{}dt\PYZcb{}}\PY{l+s+s2}{\PYZob{}}\PY{l+s+s2}{1+t\PYZca{}2\PYZcb{}\PYZcb{}\PYZdl{}}\PY{l+s+s2}{\PYZdq{}}\PY{p}{,} \PY{n}{fontsize} \PY{o}{=} \PY{l+m+mi}{26}\PY{p}{)}
        \PY{n}{ax}\PY{o}{.}\PY{n}{set\PYZus{}xlabel}\PY{p}{(}\PY{l+s+sa}{r}\PY{l+s+s2}{\PYZdq{}}\PY{l+s+s2}{\PYZdl{}t\PYZdl{}}\PY{l+s+s2}{\PYZdq{}}\PY{p}{)}
        \PY{n}{ax}\PY{o}{.}\PY{n}{legend}\PY{p}{(}\PY{p}{[}\PY{l+s+s2}{\PYZdq{}}\PY{l+s+s2}{\PYZdl{}tan\PYZca{}}\PY{l+s+s2}{\PYZob{}}\PY{l+s+s2}{\PYZhy{}1\PYZcb{}(t)\PYZdl{}}\PY{l+s+s2}{\PYZdq{}}\PY{p}{,}\PY{l+s+s2}{\PYZdq{}}\PY{l+s+s2}{quad}\PY{l+s+s2}{\PYZdq{}}\PY{p}{]}\PY{p}{,}\PY{n}{loc}\PY{o}{=}\PY{l+m+mi}{0}\PY{p}{)}
        \PY{n}{show}\PY{p}{(}\PY{p}{)}
\end{Verbatim}


    \begin{center}
    \adjustimage{max size={0.9\linewidth}{0.9\paperheight}}{Assignment2_files/Assignment2_14_0.png}
    \end{center}
    { \hspace*{\fill} \\}
    
    The error is plotted below on a semi-log axis:

    \begin{Verbatim}[commandchars=\\\{\}]
{\color{incolor}In [{\color{incolor}10}]:} \PY{n}{semilogy}\PY{p}{(}\PY{p}{)}
         \PY{n}{scatter}\PY{p}{(}\PY{n}{x}\PY{p}{,}\PY{n}{errors}\PY{p}{,}\PY{n}{s}\PY{o}{=}\PY{l+m+mi}{50}\PY{p}{)}
         \PY{n}{title}\PY{p}{(}\PY{l+s+sa}{r}\PY{l+s+s2}{\PYZdq{}}\PY{l+s+s2}{Error (returned by quad) in \PYZdl{}}\PY{l+s+s2}{\PYZbs{}}\PY{l+s+s2}{int\PYZus{}0\PYZca{}a}\PY{l+s+s2}{\PYZob{}}\PY{l+s+s2}{\PYZbs{}}\PY{l+s+s2}{frac}\PY{l+s+si}{\PYZob{}dt\PYZcb{}}\PY{l+s+s2}{\PYZob{}}\PY{l+s+s2}{1+t\PYZca{}2\PYZcb{}\PYZcb{}\PYZdl{}}\PY{l+s+s2}{\PYZdq{}}\PY{p}{,} \PY{n}{fontsize}\PY{o}{=}\PY{l+m+mi}{26}\PY{p}{)}
         \PY{n}{ylabel}\PY{p}{(}\PY{l+s+s2}{\PYZdq{}}\PY{l+s+s2}{Error}\PY{l+s+s2}{\PYZdq{}}\PY{p}{)}
         \PY{n}{xlabel}\PY{p}{(}\PY{l+s+s2}{\PYZdq{}}\PY{l+s+s2}{\PYZdl{}t\PYZdl{}}\PY{l+s+s2}{\PYZdq{}}\PY{p}{)}
         \PY{n}{show}\PY{p}{(}\PY{p}{)}
\end{Verbatim}


    \begin{center}
    \adjustimage{max size={0.9\linewidth}{0.9\paperheight}}{Assignment2_files/Assignment2_16_0.png}
    \end{center}
    { \hspace*{\fill} \\}
    
    \begin{Verbatim}[commandchars=\\\{\}]
{\color{incolor}In [{\color{incolor}11}]:} \PY{n}{semilogy}\PY{p}{(}\PY{p}{)}
         \PY{n}{title}\PY{p}{(}\PY{l+s+sa}{r}\PY{l+s+s2}{\PYZdq{}}\PY{l+s+s2}{Error (with arctan) in \PYZdl{}}\PY{l+s+s2}{\PYZbs{}}\PY{l+s+s2}{int\PYZus{}0\PYZca{}a}\PY{l+s+s2}{\PYZob{}}\PY{l+s+s2}{\PYZbs{}}\PY{l+s+s2}{frac}\PY{l+s+si}{\PYZob{}dt\PYZcb{}}\PY{l+s+s2}{\PYZob{}}\PY{l+s+s2}{1+t\PYZca{}2\PYZcb{}\PYZcb{}\PYZdl{}}\PY{l+s+s2}{\PYZdq{}}\PY{p}{,} \PY{n}{fontsize}\PY{o}{=}\PY{l+m+mi}{26}\PY{p}{)}
         \PY{n}{ylabel}\PY{p}{(}\PY{l+s+s2}{\PYZdq{}}\PY{l+s+s2}{Error}\PY{l+s+s2}{\PYZdq{}}\PY{p}{)}
         \PY{n}{xlabel}\PY{p}{(}\PY{l+s+s2}{\PYZdq{}}\PY{l+s+s2}{\PYZdl{}t\PYZdl{}}\PY{l+s+s2}{\PYZdq{}}\PY{p}{)}
         \PY{n}{scatter}\PY{p}{(}\PY{n}{x}\PY{p}{,}\PY{n+nb}{abs}\PY{p}{(}\PY{n}{integrals}\PY{o}{\PYZhy{}}\PY{n}{arctan}\PY{p}{(}\PY{n}{x}\PY{p}{)}\PY{p}{)}\PY{p}{,}\PY{n}{s}\PY{o}{=}\PY{l+m+mi}{50}\PY{p}{,}\PY{n}{color}\PY{o}{=}\PY{l+s+s1}{\PYZsq{}}\PY{l+s+s1}{red}\PY{l+s+s1}{\PYZsq{}}\PY{p}{)}
         \PY{n}{show}\PY{p}{(}\PY{p}{)}
\end{Verbatim}


    \begin{center}
    \adjustimage{max size={0.9\linewidth}{0.9\paperheight}}{Assignment2_files/Assignment2_17_0.png}
    \end{center}
    { \hspace*{\fill} \\}
    
    \section{Part 5}\label{part-5}

\subsection{Question (a)}\label{question-a}

The trapezoidal integration method is implemented using a python loop
below. The running sum is manually calculated, and is reused to find the
value of the integral at each time step.

    \begin{Verbatim}[commandchars=\\\{\}]
{\color{incolor}In [{\color{incolor}12}]:} \PY{k}{def} \PY{n+nf}{trapForLoop}\PY{p}{(}\PY{n}{f}\PY{p}{,}\PY{n}{start}\PY{p}{,}\PY{n}{end}\PY{p}{,}\PY{n}{h}\PY{p}{)}\PY{p}{:}
             \PY{l+s+sd}{\PYZdq{}\PYZdq{}\PYZdq{}}
         \PY{l+s+sd}{    Integrates f from start to end using the Trapezoidal method with}
         \PY{l+s+sd}{    step size h. }
         \PY{l+s+sd}{    }
         \PY{l+s+sd}{    Returns a vector of values corresponding to the integral at each}
         \PY{l+s+sd}{    time step.}
         \PY{l+s+sd}{    }
         \PY{l+s+sd}{    (using a loop)}
         \PY{l+s+sd}{    \PYZdq{}\PYZdq{}\PYZdq{}}
             \PY{n}{x} \PY{o}{=} \PY{n}{start}
             \PY{n}{val}\PY{o}{=}\PY{l+m+mi}{0}
             \PY{n}{s}\PY{o}{=}\PY{l+m+mi}{0} \PY{c+c1}{\PYZsh{} cumulative sum}
             \PY{n}{f0} \PY{o}{=} \PY{n}{f}\PY{p}{(}\PY{n}{start}\PY{p}{)} \PY{c+c1}{\PYZsh{} first function value}
             \PY{n}{integrals} \PY{o}{=} \PY{p}{[}\PY{p}{]} \PY{c+c1}{\PYZsh{} stores integral at each time step}
             
             \PY{c+c1}{\PYZsh{} python loop implementation of the trapezoidal method}
             \PY{k}{while} \PY{n}{x} \PY{o}{\PYZlt{}}\PY{o}{=} \PY{n}{end}\PY{p}{:}
                 \PY{n}{val} \PY{o}{=} \PY{n}{f}\PY{p}{(}\PY{n}{x}\PY{p}{)}
                 \PY{n}{s} \PY{o}{+}\PY{o}{=} \PY{n}{val}
                 \PY{n}{x} \PY{o}{+}\PY{o}{=} \PY{n}{h}
                 
                 \PY{n}{integrals}\PY{o}{.}\PY{n}{append}\PY{p}{(}\PY{p}{(}\PY{n}{s}\PY{o}{\PYZhy{}}\PY{l+m+mf}{0.5}\PY{o}{*}\PY{p}{(}\PY{n}{f0}\PY{o}{+}\PY{n}{val}\PY{p}{)}\PY{p}{)}\PY{o}{*}\PY{n}{h}\PY{p}{)}
                 
             \PY{k}{return} \PY{n}{integrals}
\end{Verbatim}


    \subsection{Question (b)}\label{question-b}

The same algorithm is implemented using vectorized numpy code. The main
computation boils down to just one line.

    \begin{Verbatim}[commandchars=\\\{\}]
{\color{incolor}In [{\color{incolor}13}]:} \PY{k}{def} \PY{n+nf}{trap}\PY{p}{(}\PY{n}{f}\PY{p}{,}\PY{n}{start}\PY{p}{,}\PY{n}{end}\PY{p}{,}\PY{n}{h}\PY{p}{)}\PY{p}{:}
             \PY{l+s+sd}{\PYZdq{}\PYZdq{}\PYZdq{}}
         \PY{l+s+sd}{    Integrates f from start to end using the Trapezoidal method with}
         \PY{l+s+sd}{    step size h. }
         \PY{l+s+sd}{    }
         \PY{l+s+sd}{    Returns a vector of values corresponding to the integral at each}
         \PY{l+s+sd}{    step.}
         \PY{l+s+sd}{    }
         \PY{l+s+sd}{    (vectorized)}
         \PY{l+s+sd}{    \PYZdq{}\PYZdq{}\PYZdq{}}
             \PY{c+c1}{\PYZsh{} Create the x values}
             \PY{n}{x} \PY{o}{=} \PY{n}{linspace}\PY{p}{(}\PY{n}{start}\PY{p}{,}\PY{n}{end}\PY{p}{,}\PY{l+m+mi}{1}\PY{o}{+}\PY{p}{(}\PY{n}{end}\PY{o}{\PYZhy{}}\PY{n}{start}\PY{p}{)}\PY{o}{/}\PY{n}{h}\PY{p}{)}
             
             \PY{c+c1}{\PYZsh{} Apply function}
             \PY{n}{y} \PY{o}{=} \PY{n}{f}\PY{p}{(}\PY{n}{x}\PY{p}{)}
             
             \PY{c+c1}{\PYZsh{} Integrate using vector operations}
             \PY{k}{return} \PY{n}{h}\PY{o}{*}\PY{p}{(}\PY{n}{cumsum}\PY{p}{(}\PY{n}{y}\PY{p}{)}\PY{o}{\PYZhy{}}\PY{l+m+mf}{0.5}\PY{o}{*}\PY{p}{(}\PY{n}{y}\PY{p}{[}\PY{l+m+mi}{0}\PY{p}{]}\PY{o}{+}\PY{n}{y}\PY{p}{)}\PY{p}{)}
\end{Verbatim}


    Let us compare the efficiency of the two approaches. 
		\begin{itemize}
			\item Clearly, the numpy version is much more concise and readable when compared to the lengthy loop implementation. 
			\item Let us compare the speeds of the two approaches, using a constant step size of, say, \(10^{-5}\).
		\end{itemize} 
    \begin{Verbatim}[commandchars=\\\{\}]
{\color{incolor}In [{\color{incolor}14}]:} \PY{n}{h} \PY{o}{=} \PY{l+m+mf}{1e\PYZhy{}5} \PY{c+c1}{\PYZsh{} step size}
\end{Verbatim}


    \begin{Verbatim}[commandchars=\\\{\}]
{\color{incolor}In [{\color{incolor}15}]:} \PY{o}{\PYZpc{}\PYZpc{}}\PY{k}{timeit}
         trapForLoop(f,0,1,h)
\end{Verbatim}


    \begin{Verbatim}[commandchars=\\\{\}]
64.9 ms ± 8.24 ms per loop (mean ± std. dev. of 7 runs, 10 loops each)

    \end{Verbatim}

    \begin{Verbatim}[commandchars=\\\{\}]
{\color{incolor}In [{\color{incolor}16}]:} \PY{o}{\PYZpc{}\PYZpc{}}\PY{k}{timeit}
         trap(f,0,1,h)
\end{Verbatim}


    \begin{Verbatim}[commandchars=\\\{\}]
1.52 ms ± 150 µs per loop (mean ± std. dev. of 7 runs, 1000 loops each)

    \end{Verbatim}

    From the above measurements, we can conclude that the vectorized version
of the algorithm performs around 40 to 50 times faster than the python
loop implementation! This underlines the importance of writing
vectorized code wherever possible.

The plot of the results of trap integration is added to the first figure
and shown below:

    \begin{Verbatim}[commandchars=\\\{\}]
{\color{incolor}In [{\color{incolor}17}]:} \PY{n}{ax}\PY{o}{.}\PY{n}{scatter}\PY{p}{(}\PY{n}{x}\PY{p}{,}\PY{n}{trap}\PY{p}{(}\PY{n}{f}\PY{p}{,}\PY{l+m+mi}{0}\PY{p}{,}\PY{l+m+mi}{5}\PY{p}{,}\PY{l+m+mf}{0.1}\PY{p}{)}\PY{p}{,}\PY{n}{marker}\PY{o}{=}\PY{l+s+s1}{\PYZsq{}}\PY{l+s+s1}{x}\PY{l+s+s1}{\PYZsq{}}\PY{p}{,}\PY{n}{s}\PY{o}{=}\PY{l+m+mi}{100}\PY{p}{,}\PY{n}{color}\PY{o}{=}\PY{l+s+s1}{\PYZsq{}}\PY{l+s+s1}{green}\PY{l+s+s1}{\PYZsq{}}\PY{p}{)}
         \PY{n}{ax}\PY{o}{.}\PY{n}{set\PYZus{}title}\PY{p}{(}\PY{l+s+s2}{\PYZdq{}}\PY{l+s+s2}{Comparing quad integration, trap integration and numpy arctan}\PY{l+s+s2}{\PYZdq{}}\PY{p}{)}
         \PY{n}{ax}\PY{o}{.}\PY{n}{legend}\PY{p}{(}\PY{p}{[}\PY{l+s+s2}{\PYZdq{}}\PY{l+s+s2}{\PYZdl{}tan\PYZca{}}\PY{l+s+s2}{\PYZob{}}\PY{l+s+s2}{\PYZhy{}1\PYZcb{}(t)\PYZdl{}}\PY{l+s+s2}{\PYZdq{}}\PY{p}{,}\PY{l+s+s2}{\PYZdq{}}\PY{l+s+s2}{quad}\PY{l+s+s2}{\PYZdq{}}\PY{p}{,}\PY{l+s+s2}{\PYZdq{}}\PY{l+s+s2}{trap}\PY{l+s+s2}{\PYZdq{}}\PY{p}{]}\PY{p}{,}\PY{n}{loc}\PY{o}{=}\PY{l+m+mi}{0}\PY{p}{)}
         \PY{n}{fig0}
\end{Verbatim}

\texttt{\color{outcolor}Out[{\color{outcolor}17}]:}
    
    \begin{center}
    \adjustimage{max size={0.9\linewidth}{0.9\paperheight}}{Assignment2_files/Assignment2_27_0.png}
    \end{center}
    { \hspace*{\fill} \\}
    

    \subsection{Question (c)}\label{question-c}

Now, let us estimate the error for different values of the step size
\(h\) and compare it with the actual error. The error is estimated by
treating the value obtained by using half the step-size as the true
value. The error between two evaluations is the maximum absolute error
over all steps of the integration.

    \begin{Verbatim}[commandchars=\\\{\}]
{\color{incolor}In [{\color{incolor}18}]:} \PY{n}{hs} \PY{o}{=} \PY{p}{[}\PY{p}{]} \PY{c+c1}{\PYZsh{} array of h values}
         \PY{n}{esterrs} \PY{o}{=} \PY{p}{[}\PY{p}{]} \PY{c+c1}{\PYZsh{} estimated errors}
         \PY{n}{acterrs} \PY{o}{=} \PY{p}{[}\PY{p}{]} \PY{c+c1}{\PYZsh{} actual errors}
         \PY{n}{h}\PY{o}{=}\PY{l+m+mf}{0.5} \PY{c+c1}{\PYZsh{} initial value of h}
         \PY{n}{tolerance} \PY{o}{=} \PY{l+m+mf}{1e\PYZhy{}8} \PY{c+c1}{\PYZsh{} desired tolerance}
         \PY{n}{e} \PY{o}{=} \PY{n}{tolerance}\PY{o}{+}\PY{l+m+mi}{1} \PY{c+c1}{\PYZsh{} initialize error greater than tolerance}
         \PY{n}{start} \PY{o}{=} \PY{l+m+mi}{0} \PY{c+c1}{\PYZsh{} start of integration}
         \PY{n}{end} \PY{o}{=} \PY{l+m+mi}{1} \PY{c+c1}{\PYZsh{} end of integration}
         
         \PY{c+c1}{\PYZsh{} A helper function to generate a linspace from start to}
         \PY{c+c1}{\PYZsh{} end with stepsize of h}
         \PY{n}{xvals} \PY{o}{=} \PY{k}{lambda} \PY{n}{h}\PY{p}{:}\PY{n}{linspace}\PY{p}{(}\PY{n}{start}\PY{p}{,}\PY{n}{end}\PY{p}{,}\PY{l+m+mi}{1}\PY{o}{+}\PY{p}{(}\PY{n}{end}\PY{o}{\PYZhy{}}\PY{n}{start}\PY{p}{)}\PY{o}{/}\PY{n}{h}\PY{p}{)}
         
         \PY{n}{ints} \PY{o}{=} \PY{l+m+mi}{0} \PY{c+c1}{\PYZsh{} estimating error for these integral values}
         
         \PY{c+c1}{\PYZsh{} the better values obtained by halving the step\PYZhy{}size}
         \PY{c+c1}{\PYZsh{} these are treated as true values when compared to \PYZsq{}ints\PYZsq{}}
         \PY{n}{better\PYZus{}ints} \PY{o}{=} \PY{n}{trap}\PY{p}{(}\PY{n}{f}\PY{p}{,}\PY{n}{start}\PY{p}{,}\PY{n}{end}\PY{p}{,}\PY{n}{h}\PY{p}{)} 
         
         \PY{c+c1}{\PYZsh{} loop until the error falls below our desired tolerance}
         \PY{k}{while} \PY{n}{e}\PY{o}{\PYZgt{}}\PY{n}{tolerance}\PY{p}{:}
             \PY{n}{ints} \PY{o}{=} \PY{n}{better\PYZus{}ints} \PY{c+c1}{\PYZsh{} update the current integral values}
             
             \PY{c+c1}{\PYZsh{} find the better values which will be treated as true values}
             \PY{n}{better\PYZus{}ints} \PY{o}{=} \PY{n}{trap}\PY{p}{(}\PY{n}{f}\PY{p}{,}\PY{n}{start}\PY{p}{,}\PY{n}{end}\PY{p}{,}\PY{n}{h}\PY{o}{/}\PY{l+m+mi}{2}\PY{p}{)}
             
             \PY{c+c1}{\PYZsh{} find the estimated error by comparing common points}
             \PY{n}{e} \PY{o}{=} \PY{n+nb}{max}\PY{p}{(}\PY{n+nb}{abs}\PY{p}{(}\PY{n}{ints}\PY{o}{\PYZhy{}}\PY{n}{better\PYZus{}ints}\PY{p}{[}\PY{p}{:}\PY{p}{:}\PY{l+m+mi}{2}\PY{p}{]}\PY{p}{)}\PY{p}{)}
             
             \PY{n}{hs}\PY{o}{.}\PY{n}{append}\PY{p}{(}\PY{n}{h}\PY{p}{)}
             \PY{n}{esterrs}\PY{o}{.}\PY{n}{append}\PY{p}{(}\PY{n}{e}\PY{p}{)}
             
             \PY{c+c1}{\PYZsh{} find the actual errors}
             \PY{n}{acterrs}\PY{o}{.}\PY{n}{append}\PY{p}{(}\PY{n+nb}{max}\PY{p}{(}\PY{n+nb}{abs}\PY{p}{(}\PY{n}{ints}\PY{o}{\PYZhy{}}\PY{n}{arctan}\PY{p}{(}\PY{n}{xvals}\PY{p}{(}\PY{n}{h}\PY{p}{)}\PY{p}{)}\PY{p}{)}\PY{p}{)}\PY{p}{)}
             
             \PY{c+c1}{\PYZsh{} update the step size}
             \PY{n}{h}\PY{o}{/}\PY{o}{=}\PY{l+m+mi}{2}
\end{Verbatim}


    Let us plot the estimated and actual errors on a log-log plot to compare
them.

    \begin{Verbatim}[commandchars=\\\{\}]
{\color{incolor}In [{\color{incolor}19}]:} \PY{n}{loglog}\PY{p}{(}\PY{p}{)}
         \PY{n}{scatter}\PY{p}{(}\PY{n}{hs}\PY{p}{,}\PY{n}{acterrs}\PY{p}{,}\PY{n}{color}\PY{o}{=}\PY{l+s+s1}{\PYZsq{}}\PY{l+s+s1}{red}\PY{l+s+s1}{\PYZsq{}}\PY{p}{,}\PY{n}{s}\PY{o}{=}\PY{l+m+mi}{50}\PY{p}{)}
         \PY{n}{scatter}\PY{p}{(}\PY{n}{hs}\PY{p}{,}\PY{n}{esterrs}\PY{p}{,}\PY{n}{color}\PY{o}{=}\PY{l+s+s1}{\PYZsq{}}\PY{l+s+s1}{green}\PY{l+s+s1}{\PYZsq{}}\PY{p}{,}\PY{n}{marker}\PY{o}{=}\PY{l+s+s1}{\PYZsq{}}\PY{l+s+s1}{+}\PY{l+s+s1}{\PYZsq{}}\PY{p}{,}\PY{n}{s}\PY{o}{=}\PY{l+m+mi}{50}\PY{p}{)}
         \PY{n}{ylabel}\PY{p}{(}\PY{l+s+s2}{\PYZdq{}}\PY{l+s+s2}{Error}\PY{l+s+s2}{\PYZdq{}}\PY{p}{)}
         \PY{n}{xlabel}\PY{p}{(}\PY{l+s+s2}{\PYZdq{}}\PY{l+s+s2}{Step size \PYZdl{}h\PYZdl{}}\PY{l+s+s2}{\PYZdq{}}\PY{p}{)}
         \PY{n}{title}\PY{p}{(}\PY{l+s+s2}{\PYZdq{}}\PY{l+s+s2}{Actual and estimated error vs step size}\PY{l+s+s2}{\PYZdq{}}\PY{p}{)}
         \PY{n}{legend}\PY{p}{(}\PY{p}{[}\PY{l+s+s2}{\PYZdq{}}\PY{l+s+s2}{Actual error}\PY{l+s+s2}{\PYZdq{}}\PY{p}{,} \PY{l+s+s2}{\PYZdq{}}\PY{l+s+s2}{Estimated error}\PY{l+s+s2}{\PYZdq{}}\PY{p}{]}\PY{p}{,}\PY{n}{loc}\PY{o}{=}\PY{l+m+mi}{2}\PY{p}{)}
         \PY{n}{show}\PY{p}{(}\PY{p}{)}
\end{Verbatim}


    \begin{center}
    \adjustimage{max size={0.9\linewidth}{0.9\paperheight}}{Assignment2_files/Assignment2_31_0.png}
    \end{center}
    { \hspace*{\fill} \\}
    
    Let's find the least squares linear regression slope of the above
log-log plot of the estimated error:

    \begin{Verbatim}[commandchars=\\\{\}]
{\color{incolor}In [{\color{incolor}20}]:} \PY{k+kn}{from} \PY{n+nn}{scipy}\PY{n+nn}{.}\PY{n+nn}{stats} \PY{k}{import} \PY{n}{linregress}
         \PY{n}{logh} \PY{o}{=} \PY{n}{log}\PY{p}{(}\PY{n}{array}\PY{p}{(}\PY{n}{hs}\PY{p}{)}\PY{p}{)}
         \PY{n}{loge} \PY{o}{=} \PY{n}{log}\PY{p}{(}\PY{n}{array}\PY{p}{(}\PY{n}{esterrs}\PY{p}{)}\PY{p}{)}
         \PY{n+nb}{print}\PY{p}{(}\PY{n}{linregress}\PY{p}{(}\PY{n}{logh}\PY{p}{,}\PY{n}{loge}\PY{p}{)}\PY{o}{.}\PY{n}{slope}\PY{p}{)}
\end{Verbatim}


    \begin{Verbatim}[commandchars=\\\{\}]
2.0004028135

    \end{Verbatim}

    \section{Conclusions}\label{conclusions}

We can make the following observations based on the plot above:

\begin{itemize}
\tightlist
\item The estimated error is not equal to the actual error. We must use the
  estimated error to determine step size when we do not know the
  analytic integral of the function we are integrating.
\item The slope of the error vs step-size plot is very close to 2. This
  means that the error in the value of the integral grows as
  \(\mathcal{O}(h^2)\), which matches with the analytical result.
\end{itemize}


    % Add a bibliography block to the postdoc
    
    
    
    \end{document}
